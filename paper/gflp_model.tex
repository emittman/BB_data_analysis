\documentclass{article}
\usepackage{amsmath}
\newcommand{\ind}{\stackrel{ind.}{\sim}}
\newcommand{\op}{\operatorname}

\begin{document}
\section{Hierarchical GLFP model}
Let $T_{d,i}$ be the time of failure for the $i^{th}$ drive of drive-model $d$.
We assume that $T_{d,1},\ldots T_{d,n_d}$ are independent and have a probability distribution with cdf given by
\begin{equation*}
P(T_{d,i}\le t) = 1 - (1-\pi_d\, F_{d1}(t))(1 - F_{d2}(t)),
\end{equation*}
where $\pi_d$ is in the interval $(0,1)$.

Our interpretation of this model is that, while all units are prone to a mode of failure which can be approximated by a distribution with cdf $F_{d2}$, a proportion $\pi$ are {\em also} defective, and thus subject to another failure mode ($F_{d1}$).

To borrow strength across models, we further model the scales, $\sigma_j$, and log-locations, $\mu_j$, of the component distributions as follows:

\begin{align*}
\sigma_{d,j} &\ind \op{Lognormal} \left( \eta_{\sigma,j}, \tau^2_{\sigma,j} \right) \mbox{ for } j=1,2\; d=1,\ldots,D\\
t_{p_j,d,j} \equiv \mu_{d,j} + \sigma_{d,j}\,\Phi^{-1}(p_j) & \ind \op{Normal} \left(\eta_{t_{p_j},j}, \tau^2_{t_{p_j},j}\right) \mbox{ for } j=1,2\; d=1,\ldots,D\\
\pi_d &\ind \op{Beta}(\alpha, \beta) \mbox{ for } d=1,\ldots,D.
\end{align*}

Here, $\Phi^{-1}$ is the quantile function of the standard log-Weibull distribution. The decision to parameterize in terms of a quantile other than the log-location parameter, $\mu = \Phi^{-1}(.5)$, is that lifetime data often features heavy right-censoring where inferences about the location parameter are extrapolations beyond the range of the data. For this data we selected $p_1=0.5,\, p_2 = 0.2$.

\subsection{Priors}
To complete the full probability model, we need to select priors such that the posterior distribution is proper. We chose informative priors; we argue that doing so helps us identify the model parameters while still allowing for substantial Bayesian learning about the hierarchical distributions.

\begin{align*}
 \eta_{\sigma ,1} & \sim \op{Normal}(0, 0.5)\\
 \eta_{\sigma ,2} & \sim \op{Normal}(-0.75, .5)\\
 \tau_{\sigma ,1} & \sim \op{Gamma}(1, 2)\\
 \tau_{\sigma ,2} & \sim \op{Gamma}(1, 2)\\
 \eta_{t_{p_1},1} & \sim \op{Normal}(7, 1)\\
 \eta_{t_{p_2},2} & \sim \op{Normal}(9, 2)\\
 \tau_{t_{p_1},1} & \sim \op{Gamma}(1, 2)\\
 \tau_{t_{p_2},2} & \sim \op{Gamma}(1, 2)
 \end{align*} 
 
\end{document}