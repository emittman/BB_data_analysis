\documentclass[12pt]{article}
\usepackage{amsmath}
\usepackage{graphicx,psfrag,epsf,float}
\graphicspath{C:/Users/Colin/Documents/GitHub/BB_data_analysis/paper}
\usepackage{enumerate}
\usepackage{natbib}
\usepackage{url} % not crucial - just used below for the URL
 
\newcommand{\myequation}{\begin{equation}}
\newcommand{\myendequation}{\end{equation}}
\let\[\myequation
\let\]\myendequation


\pdfminorversion=4
% NOTE: To produce blinded version, replace "0" with "1" below.
\newcommand{\blind}{0}

% DON'T change margins - should be 1 inch all around.
\addtolength{\oddsidemargin}{-.5in}%
\addtolength{\evensidemargin}{-.5in}%
\addtolength{\textwidth}{1in}%
\addtolength{\textheight}{1.3in}%
\addtolength{\topmargin}{-.8in}%


\begin{document}

%\bibliographystyle{natbib}

\def\spacingset#1{\renewcommand{\baselinestretch}%
{#1}\small\normalsize} \spacingset{1}


%%%%%%%%%%%%%%%%%%%%%%%%%%%%%%%%%%%%%%%%%%%%%%%%%%%%%%%%%%%%%%%%%%%%%%%%%%%%%%

\if0\blind
{
  \title{\bf A Hierarchical Bayesian Approach for Modeling Infant-Mortality and Wearout Failure Modes}
  \author{Eric Mittman 1\thanks{
    The authors gratefully acknowledge Bill Meeker for his comments and suggestions}\hspace{.2cm}\\
    Department of Statistics, Iowa State University\\
    and \\
    Colin Lewis-Beck \\
    Department of Statistics, Iowa State University}
  \maketitle
} \fi

\if1\blind
{
  \bigskip
  \bigskip
  \bigskip
  \begin{center}
    {\LARGE\bf Title}
\end{center}
  \medskip
} \fi

\bigskip
\begin{abstract}
The text of your abstract.  100 or fewer words.
\end{abstract}

\noindent%
{\it Keywords:}  3 to 6 keywords, (don't reuse words appearing in title)
\vfill

\newpage
\spacingset{1.45} % DON'T change the spacing!
\section{Introduction}
We present an approach to modeling failure data with heavy censoring, multiple censoring and multiple truncation. We present the generalized limited failure population model (GFLP) of Chan and Meeker which provides a generative model for failure distributions characterized by a "bathtub" hazard function (citation). This model may provide an adequate fit to failures of complex systems by differentiating between early and late types of failures as well as the propensity toward early failure. We discuss the case where the data are grouped, suggesting that we might expand the model, accounting for differences among groups. We show that by modeling the data hierarchically, we can deal with groups of various sizes by borrowing information across groups. We illustrate our approach on a large set of failure data for hard disk drives made available by Backblaze (citation). 

\subsection{Background}
Products can frequently fail due to more than one cause.  For example, there are many parts in a washing machine that can break and cause the whole machine to fail.  Decorative tree lighting can fail if one bulb in the chain burns out.  The general name for such products is a series system where the lifetime of the product is the minimum failure time across $s$ different components or risks \cite{nelson}.  A common assumption is the times to failure for each risk, $s$, are statistically independent.  Thus, the overall reliability of a unit can be modeled using the product rule across all $s$ risks.  The GFLP model is a special case of the $s$ independent competing risks model with 2 modes of failure: defects and wearout.  Sometimes the cause of the failure is know; but more often the observations do not contain any information about the failure mode \cite{chan}.  The GFLP model was first used to describe the early and late failure modes of CB radios.  No failure information was available for this data, which made fitting the model a challenge using maximum likelihood estimation.  Other authors, for example, Basu et al., used a Bayesian approach to fit the GFLP model to masked failure data from an engineering system \cite{basu}.  More recently, Ranjan et al. considered a competing risk model for infant mortality and wearout as a mixture of Weibull and exponential failure distributions \cite{ranjan}.  
 
\subsection{Motivation}
The goal of this paper is to model, and compare, the lifetime distribution of the various hard drive brands using parametric lifetime models.  There are numerous features of the data, however, that make standard estimation approaches problematic.  The first issue is many hard drives have been running for quite some time before entering the sample.  Even hard drives starting after 2013 appear in the data after hundreds of hours of operation.  Also, although we have four years of data, hard drives rarely fail so many units are still in service.  Thus, we have both left truncation and right censoring.  The other challenge is each hard drive model varies in number units in operation and total time on test.  Some models have hundreds of units in service; others have as few as two.  Combined with the truncation this makes stable maximum likelihood (ML) estimates difficult to obtain for many drive models. \\

Another issue is many product populations contain a mixture of defective and nondefective units.  The hazard function for this type of population is often described as a bathtub curve: the beginning of the curve corresponds to defective units failing early, followed by a constant hazard, and then and upswing as units fail from wearout.  Ignoring this hazard structure, which is known to exist for computer disk drives, could lead to spurious inference when comparing the reliability of hard drive brand models \cite{chan}.  A model that accounts for multiple failure modes is the general limited failure population model (GFLP) that combines different failure time distributions; for example, one failure mode for defective units (infant mortality) and a second mode for wearout \cite{chan}.  Unfortunately, as often the case with lifetime data, the true causes of failure for the hard drives are unknown, which presents challenges for parameter identification in the GFLP model, as discussed in Chan and Meeker (1999).  Therefore, we propose to fit the GFLP model using a hierarchical Bayesian approach.  With over 60 brand drive models in testing a hierarchical framework is advantageous as it pools information from across brands to get more precise estimates of the individual parameters.  This is especially helpful for the Backblaze data as there are many drive models with as few as one or two failures.

\subsection{Overview}
The structure of the paper is as follows.  Section 1 presents the Backblaze data and the unique modeling challenges it presents, such as multiple failure modes, right censoring, and left truncation.  Section 2 introduces the GFLP model as a mixture of two Weibull distributions.  Using the Backblaze data, in Section 3 we fit a 5 parameter GFLP model that allows for separate Weibull terms for each failure mode and estimates $p$, the proportion of units subject to infant mortality.  In Section 4 we discuss computational and modeling issues that sugguests a final model with a common Weibull distribution for defective units and individual Weibull distributions for the wearout failure mode.  In Section 5 we compare hard drive brand models based on our final model.  In Section 6  we examine the posterior distributions of the model parameters to assess goodness of fit for the hierarchical model.  Finally, in Section 7 we review the GFLP model and propose future extensions.

\section{Data}
Backblaze is a company that offers backup storage to protect against hard drive failure.  While selling storage space is Backblaze's main business, since 2013 it has been collecting data on hard drives operating at its facility.  The purpose is to provide consumers and businesses with reliability information on different hard drive brands and models.  The hard drives continuously spin in controlled storage pods where they run until failure.  When a hard drive fails it is permanently removed, and new hard drives are regularly added to the storage pods.  In addition, the number of storage pods is increasing as Backblaze adds new hard drive brands to the sample.  Every quarter Backblaze makes its data publicly available through their website \cite{backblaze}. In addition, Backblaze publishes summary statistics of the different hard drive models currently operating.  No other analysis or modeling of the failure data is provided; however, Backblaze does encourage the public to further analyze its data, which for this paper goes through the first quarter of 2016.

As of the first quarter of 2016, Backblaze was collecting data on 63 different hard drive models.  Some drive models have been running since 2013, while others have been added at a later date.  There are a total of 75,297 hard drives running.  The distribution of drives by model varies: some models have only 1 drive in testing; the maximum number of drives running for a specific drive model is 35,860.  We can also look at the distribution of total number of failures and total time on test across all drive models.  As seen below, some drive models contain a lot of information, while others have few failures or a little time in operation. [INSERT FIGURE 1]. \\

Probability plotting is an effective method to check the adequacy of various parametric models to the data.  Identifying whether failure data follows a specific distribution is difficult to do by eye: especially when looking at pdfs or other non-linear plots.  However, if we can linearize the cumulative distribution function it is easier to visually assess a distributional goodness of fit; if a model is appropriate, a non parametric estimate of $\hat{F(t)}$ graphed on linarized probability scales should approximately follow a straight line. \\

For the hard drive data we first estimate the empirical cdf for each model using the Kaplan-Meier estimator \cite{kaplan}.  With left truncation, however, the Kaplan-Meier estimate, $\widehat{F(t)_{KM}}$, is conditional based on hard drive $i$ surviving up to time $\tau_i^L$ where $L$ is the amount of time the hard drive was running before Backblaze starting monitoring the drive.  To get the unconditional distribution we correct the non-parametric estimates using a parametric adjustment outlined by Turnbull, and given in more detail by Meeker and Escobar \cite{turnbull,meeker}.  For each hard drive model we select $\tau_{\text{min}}^L$, the smallest left truncated time in the sample.  Then, using the parametric estimates of the parameters, we calculate $Pr(T>\tau_\text{min}^L)$ the probability a hard drive has survived up to $\tau_{\text{min}}^L$.  We then apply this correction to the non parametric estimates, which results in the unconditional distribution of time to failure.\\

In Figure 2 we plot the Kaplan-Meier adjusted cdf for hard drive model X on Weibull paper.  Each point on the plot corresponds to a hard drive failure.  Censored drives are not plotted.  As mentioned in the introduction, the population of hard drives exhibits two primary failure modes.  One mode is a result of manufacturing defects, which cause early failures, known as infant mortality.  The second mode is non-defective hard drives that eventually fail due to wearout.   Evidence of at least two failure modes is seen in the Kaplan Meier plot with a kink occurring around hour Z.  Therefore, fitting a single Weibull model would not be flexible enough to model the failure distribution.     [INSERT FIGURE, Do we Want to Put Greenwood Bands on Too?]


\section{GFLP Model}
\subsection{Model}
Write out the full Bayes hierarchical model for the GFLP model
\subsubsection{Varying proportion defective}
\subsubsection{Varying wearout mode}
\subsubsection{Varying defective failure model}


\subsection{Computation}
Discuss the MCMC approach, RStan, etc.

\section{Data analysis}
\subsection{Results}
Show some of the fits of the GFLP Model.  Also present the unpooled output for a few models.  Show parameter estimates with uncertainty. Mention some convergence statistics for the MCMC. 
\subsection{collapsing $\mu_1, \sigma_1$}
Based on posterior plots, we might fit a simpler model...
\subsection{Brand Comparisons}
How can we use this model to compare brands?  Present comparisons of Quantiles, other parameters.  Highlight how this model could be used in an applied sense. Quantiles by brand, etc.
\section{Concluding Remarks and Extensions}
Review the advantages of fitting the GFLP model and offer future ideas.  




    


\bigskip
\begin{center}
{\large\bf SUPPLEMENTARY MATERIAL}
\end{center}

\begin{description}

\item Put R Stan code here

\end{description}

\bibliographystyle{plain}
\bibliography{sample}

\end{document}
